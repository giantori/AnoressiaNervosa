% Options for packages loaded elsewhere
\PassOptionsToPackage{unicode}{hyperref}
\PassOptionsToPackage{hyphens}{url}
%
\documentclass[
]{article}
\usepackage{amsmath,amssymb}
\usepackage{iftex}
\ifPDFTeX
  \usepackage[T1]{fontenc}
  \usepackage[utf8]{inputenc}
  \usepackage{textcomp} % provide euro and other symbols
\else % if luatex or xetex
  \usepackage{unicode-math} % this also loads fontspec
  \defaultfontfeatures{Scale=MatchLowercase}
  \defaultfontfeatures[\rmfamily]{Ligatures=TeX,Scale=1}
\fi
\usepackage{lmodern}
\ifPDFTeX\else
  % xetex/luatex font selection
\fi
% Use upquote if available, for straight quotes in verbatim environments
\IfFileExists{upquote.sty}{\usepackage{upquote}}{}
\IfFileExists{microtype.sty}{% use microtype if available
  \usepackage[]{microtype}
  \UseMicrotypeSet[protrusion]{basicmath} % disable protrusion for tt fonts
}{}
\makeatletter
\@ifundefined{KOMAClassName}{% if non-KOMA class
  \IfFileExists{parskip.sty}{%
    \usepackage{parskip}
  }{% else
    \setlength{\parindent}{0pt}
    \setlength{\parskip}{6pt plus 2pt minus 1pt}}
}{% if KOMA class
  \KOMAoptions{parskip=half}}
\makeatother
\usepackage{xcolor}
\usepackage[margin=1in]{geometry}
\usepackage{longtable,booktabs,array}
\usepackage{calc} % for calculating minipage widths
% Correct order of tables after \paragraph or \subparagraph
\usepackage{etoolbox}
\makeatletter
\patchcmd\longtable{\par}{\if@noskipsec\mbox{}\fi\par}{}{}
\makeatother
% Allow footnotes in longtable head/foot
\IfFileExists{footnotehyper.sty}{\usepackage{footnotehyper}}{\usepackage{footnote}}
\makesavenoteenv{longtable}
\usepackage{graphicx}
\makeatletter
\newsavebox\pandoc@box
\newcommand*\pandocbounded[1]{% scales image to fit in text height/width
  \sbox\pandoc@box{#1}%
  \Gscale@div\@tempa{\textheight}{\dimexpr\ht\pandoc@box+\dp\pandoc@box\relax}%
  \Gscale@div\@tempb{\linewidth}{\wd\pandoc@box}%
  \ifdim\@tempb\p@<\@tempa\p@\let\@tempa\@tempb\fi% select the smaller of both
  \ifdim\@tempa\p@<\p@\scalebox{\@tempa}{\usebox\pandoc@box}%
  \else\usebox{\pandoc@box}%
  \fi%
}
% Set default figure placement to htbp
\def\fps@figure{htbp}
\makeatother
\setlength{\emergencystretch}{3em} % prevent overfull lines
\providecommand{\tightlist}{%
  \setlength{\itemsep}{0pt}\setlength{\parskip}{0pt}}
\setcounter{secnumdepth}{-\maxdimen} % remove section numbering
\usepackage{bookmark}
\IfFileExists{xurl.sty}{\usepackage{xurl}}{} % add URL line breaks if available
\urlstyle{same}
\hypersetup{
  pdftitle={Anoressia Nervosa},
  pdfauthor={Arianna Ruggiero, Gianluca Tori},
  hidelinks,
  pdfcreator={LaTeX via pandoc}}

\title{Anoressia Nervosa}
\author{Arianna Ruggiero, Gianluca Tori}
\date{14/6/2025}

\begin{document}
\maketitle

\paragraph{Contesto}\label{contesto}

Si vuole valutare se è possibile individuare i segni precoci di
disfunzione articolare o muscolare caratteristici delle fasi iniziali di
anoressia nervosa in strutture che non consentono le misurazioni di
prova fisica utilizzando i parametri bioimpedenziometrici e altri marker
clinici indiretti. Un'associazione tra i parametri di forza fisica e
funzionalità motoria e i parametri BIA e marker clinci indiretti sarebbe
potenzialmente utile nei centri sprovvisti di ergometria. Si vuole
quindi dimostrare che BIA e altri marker clinici indiretti e i parametri
di forza muscolare e funzionalità motoria migliorano o peggiorano in
parallelo e identificare quali marker clinici indiretti anticipano (o
riflettono) il recupero funzionale.

Il nostro campione ha un totale di 23 osservazioni e 123 variabili.
Numero di pazienti con anoressia nervosa: 21 Numero di pazienti con
bulimia nervosa: 2 Si rimuovono le seguenti colonne, al fine di non
avere informazioni ridondanti e inoltre alcune rappresentano solo un
calcolo intermedio usato per derivare altri indici: h2 --\textgreater{}
altezza al quadrato data ricovero data\_T0 data\_T1 data\_T2 FFM, FFMp →
si preferisce tenere FFMI perchè è più accurato, corregge in base alla
statura, distingue soggetti apparentemente normopeso ma con massa magra
ridotta

Si crea invece la variabile relativa al tempo, con modalità T0, T1 e T2
per il formato lungo

\begin{longtable}[]{@{}
  >{\raggedright\arraybackslash}p{(\linewidth - 2\tabcolsep) * \real{0.5500}}
  >{\raggedright\arraybackslash}p{(\linewidth - 2\tabcolsep) * \real{0.4500}}@{}}
\toprule\noalign{}
\begin{minipage}[b]{\linewidth}\raggedright
Tipologia variabile
\end{minipage} & \begin{minipage}[b]{\linewidth}\raggedright
variabile
\end{minipage} \\
\midrule\noalign{}
\endhead
\bottomrule\noalign{}
\endlastfoot
\textbf{covariate fisse nel tempo} & patologia \\
~ & sintomo 1 \\
~ & sintomo 2 \\
~ & arto dominante \\
~ & altezza \\
~ & sesso \\
~ & età \\
\textbf{covariate tempo dipendenti} & peso\_T0, peso\_T1, peso\_T2 \\
~ & BMI\_T0, BMI\_T1, BMI\_T2 \\
~ & mkcal\_T0, mkcal\_T1, mkcal\_T2 \\
~ & variazmenu\_T0, variazmenu\_T1, variazmenu\_T2 \\
\textbf{Marker clinici indiretti a T0, T1, T2} & rq → quoziente
respiratorio \\
~ & rmr → metabolismo basale o a riposo, quantità minima di energia di
cui l'organismo ha bisogno per rimanere in vita \\
~ & vo2 max → volume di O2 consumata in un minuto durante l'attività
fisica molto intensa \\
~ & Rx\_ohm → resistenza \\
~ & Xc\_ohm → reattanza \\
~ & FM → massa grassa \\
~ & FMp → \% massa grassa \\
~ & FFMI → indice di massa magra normalizzato \\
~ & TBW → acqua corporea totale \\
~ & TBWp → \% acqua corporea totale \\
~ & ECW → acqua extracellulare \\
~ & ECWp → \% acqua extracellulare \\
~ & ICW acqua intracellulare \\
~ & ICWp \% acqua intracellulare \\
~ & BCM → massa cellulare corporea \\
~ & pha → integrità delle membrane cellulari \\
~ & BCMI → massa cellulare corporea in rapporto alla statura \\
\textbf{parametri di forza fisica e funzionalità motoria a T0, T1, T2} &
sx1, sx2, sx3, dx1, dx2, dx3 → 3 prove di forza arti superiori \\
~ & mediasx, mediadx → media prove di forza arti superiori \\
~ & dssx, dsdx → dev standard prove di forza arti superiori \\
~ & situptest \\
~ & squattest \\
~ & chairstandtest \\
~ & sitreachtest \\
\end{longtable}

Il campione ha un totale di 23 osservazioni e 116 variabili.

\paragraph{Letteratura}\label{letteratura}

\begin{itemize}
\tightlist
\item
  Un \textbf{valore basso} di pha riflette un cattivo stato nutrizionale
  e muscolare, tipico dei primi stati di anoressia, un angolo di fase
  elevato invece è associato ad una maggiore massa muscolare e ad una
  migliore salute cellulare
\item
  Un \textbf{rapporto elevato} di ECW/ICW indica ritenzione cellulare e
  quindi segno di squilibrio
\item
  Un \textbf{valore basso} di FFMI indica una minore resistenza
  muscolare e forza fisica
\item
  Un \textbf{valore basso} di BCM indica il rischio di debolezza
  muscolare
\item
  Un \textbf{valore basso} di RMR indica un adattamento metabolico da
  restrizione calorica cronica
\item
  Un \textbf{valore basso} di VO2 indica una ridotta capacità
  dell'organismo di utilizzare ossigeno per l'attività fisica
\end{itemize}

\subsubsection{Analisi esplorativa}\label{analisi-esplorativa}

\pandocbounded{\includegraphics[keepaspectratio]{prova_files/figure-latex/grafico-sxdx-1.pdf}}

Il grafico sopra mostra l'andamento della forza nel tempo (3 punti
temporali) su entrambi i lati (dx = destro, sx = sinistro) per ciascun
soggetto. Per ogni soggetto si hanno due pannelli, uno per ogni lato del
corpo. Alcuni soggetti mostrano aumenti marcati nel tempo, altri
mostrano una forza costante nel tempo. In alcuni casi, un lato è
sistematicamente più forte o mostra un andamento diverso.

Il grafico sottostante mostra la forza media per ciascun soggetto,
distinta per lato del corpo. In molti pazienti il lato destro (dx)
presenta valori medi superiori al lato sinistro, indicando una possibile
dominanza funzionale e si può anche osservare un'ampia variabilità tra
soggetti.

\pandocbounded{\includegraphics[keepaspectratio]{prova_files/figure-latex/grafico-latopiùforte-1.pdf}}

\subsubsection{Lattice plot per i test di performance
fisica}\label{lattice-plot-per-i-test-di-performance-fisica}

\pandocbounded{\includegraphics[keepaspectratio]{prova_files/figure-latex/grafico-situptest-1.pdf}}

\pandocbounded{\includegraphics[keepaspectratio]{prova_files/figure-latex/grafico-squattest-1.pdf}}

\pandocbounded{\includegraphics[keepaspectratio]{prova_files/figure-latex/grafico-chairstandtest-1.pdf}}

\pandocbounded{\includegraphics[keepaspectratio]{prova_files/figure-latex/grafico-sitreachtest-1.pdf}}

\subsection{Analisi delle componenti
principali}\label{analisi-delle-componenti-principali}

\pandocbounded{\includegraphics[keepaspectratio]{prova_files/figure-latex/grafico-pheatmap-1.pdf}}
\pandocbounded{\includegraphics[keepaspectratio]{prova_files/figure-latex/grafico-pheatmap-2.pdf}}

\end{document}
